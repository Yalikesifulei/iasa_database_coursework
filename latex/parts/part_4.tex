\newpage
\chapter{ВИСНОВОК}
В ході виконання курсової роботи було розроблено інформаційну систему для закладу
вищої освіти, робота якої була перевірена на тестових даних.
Всі функціональні вимоги до системи були виконані, необхідна функціональність реалізована за рахунок 
використання процедур та запитів.
Усі процедури були реалізовані таким чином, що передавати можна довільні комбінації
допустимих параметрів, що забезпечило гнучкість використання системи.
До переваг розробленої системи можна віднести зручний веб-інтерфейс, який значно спрощує
роботу з системою для користувачів, що не знайомі з мовою запитів MySQL та середовищем
MySQL Workbench. Також, у системи доволі мало технічних вимог.
Перспективами розвитку такої системи є додавання можливості адміністрування БД за допомогою
веб-інтерфейсу, оскільки наразі в ньому немає підтримки додавання або видалення даних з таблиць,
а також -- підтримка віддаленого доступу до системи, оскільки зараз сервер та сама система
мають знаходитися на ПК користувача.
Повний код роботи можна переглянути на GitHub за посиланням
{\small \url{https://github.com/Yalikesifulei/iasa_database_coursework}}.