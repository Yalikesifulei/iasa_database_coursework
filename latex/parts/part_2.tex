% !TEX root = ../main.tex
\newpage
\chapter{Архітектура та інформаційне забезпечення БД}
\section{Аналіз функціонування та організаційні засади підприємства}
До основних функціональних завдань ЗВО належить планування занять та формування їх розкладу,
облік викладачів та студентів, контроль успішності студентів.
Для зберігання необхідних даних доцільно сформувати такі таблиці:
<<Факультети>>, <<Кафедри>>, <<Групи>>, <<Студенти>>, <<Викладачі>>, 
<<Дисципліни>>, <<Розклад>> та <<Сесія>>. На кожному факультеті є кафедри,
за якими закріплені викладачі та навчальні групи студентів. В таблиці <<Дисципліни>>
має бути загальний перелік дисциплін, а вже в таблиці <<Розклад>> -- розподіл
дисциплін за групами та викладачами.

\section{Проектування структури бази даних}
Структуру таблиць БД та зв'язки між ними зображено на EER-діаграмі:
\begin{figure}[h]
    \centering
    \includegraphics[scale=0.8]{pics/eer.png}
    \caption{EER-діаграма}
\end{figure}