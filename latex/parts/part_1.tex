% !TEX root = ../main.tex
\newpage
\chapter{ПОСТАНОВКА ЗАДАЧІ}
Студенти, організовані в групи, які навчаються на одному з 
факультетів, очолюваному деканатом, в функції якого входить 
контроль навчального процесу. У навчальному процесі беруть 
участь викладачі кафедр, адміністративно відносяться до одного 
з факультетів. Викладачі поділяються на такі категорії: асистенти, 
викладачі, старші викладачі, доценти, професори. Асистенти і 
викладачі можуть навчатися в аспірантурі, ст. викладачі, доценти, 
можуть очолювати наукові теми, професора -- наукові напрямки. 
Викладачі будь-якої категорії свого часу могли захистити 
кандидатську, а доценти і професори і докторську дисертацію, 
при цьому викладачі можуть займати посади доцента і професора 
тільки, якщо вони мають відповідно звання доцента і професора.
Навчальний процес регламентується навчальним планом, в якому 
вказується, які навчальні дисципліни на яких курсах і у яких 
семестрах читаються для студентів кожного року набору, із 
зазначенням кількості годин на кожен вид занять з дисципліни 
(види занять: лекції, семінари, лабораторні роботи, консультації, 
курсові роботи, і т.д.) і форми контролю (залік, іспит). 
Перед початком навчального семестру деканати роздають на кафедри 
навчальні доручення, в яких вказуються будь кафедри (не обов'язково
пов'язані з цим факультетом), які дисципліни і для яких груп
повинні вести в черговому семестрі. Керуючись ними, на кафедрах 
здійснюється розподіл навантаження, при цьому по одній дисципліні 
в одній групі різні види занять можуть вести один або кілька різних
викладачів кафедри (з урахуванням категорії викладачів, наприклад, 
асистент не може читати лекції, а професор ніколи не буде проводити
лабораторні роботи). Викладач може вести заняття по одній або декількох
дисциплінах для студентів як свого, так і інших факультетів. Відомості
про проведені іспитах і заліках збираються деканатом.
Після закінчення навчання студент виконує дипломну роботу, 
керівником якої є викладач кафедри, що відноситься до того 
ж факультету, де навчається студент, при цьому викладач може 
керувати кількома студентами. 

\label{task_list}Види запитів в інформаційний системі:
\begin{enumerate}
    \makeatletter
    \@for\num:={1,2,3,4,5,6,7,8,9,10,11,12,13}\do{
        \item \input{query_descr/\num.txt}
    }
    \makeatother
\end{enumerate}