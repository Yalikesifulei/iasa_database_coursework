% !TEX root = ../main.tex
\newpage
\chapter{Постановка задачі}
Студенти, організовані в групи, які навчаються на одному з 
факультетів, очолюваному деканатом, в функції якого входить 
контроль навчального процесу. У навчальному процесі беруть 
участь викладачі кафедр, адміністративно відносяться до одного 
з факультетів. Викладачі поділяються на такі категорії: асистенти, 
викладачі, старші викладачі, доценти, професори. Асистенти і 
викладачі можуть навчатися в аспірантурі, ст. викладачі, доценти, 
можуть очолювати наукові теми, професора -- наукові напрямки. 
Викладачі будь-якої категорії свого часу могли захистити 
кандидатську, а доценти і професори і докторську дисертацію, 
при цьому викладачі можуть займати посади доцента і професора 
тільки, якщо вони мають відповідно звання доцента і професора.
Навчальний процес регламентується навчальним планом, в якому 
вказується, які навчальні дисципліни на яких курсах і у яких 
семестрах читаються для студентів кожного року набору, із 
зазначенням кількості годин на кожен вид занять з дисципліни 
(види занять: лекції, семінари, лабораторні роботи, консультації, 
курсові роботи, і т.д.) і форми контролю (залік, іспит). 
Перед початком навчального семестру деканати роздають на кафедри 
навчальні доручення, в яких вказуються будь кафедри (не обов'язково
пов'язані з цим факультетом), які дисципліни і для яких груп
повинні вести в черговому семестрі. Керуючись ними, на кафедрах 
здійснюється розподіл навантаження, при цьому по одній дисципліні 
в одній групі різні види занять можуть вести один або кілька різних
викладачів кафедри (з урахуванням категорії викладачів, наприклад, 
асистент не може читати лекції, а професор ніколи не буде проводити
лабораторні роботи). Викладач може вести заняття по одній або декількох
дисциплінах для студентів як свого, так і інших факультетів. Відомості
про проведені іспитах і заліках збираються деканатом.
Після закінчення навчання студент виконує дипломну роботу, 
керівником якої є викладач кафедри, що відноситься до того 
ж факультету, де навчається студент, при цьому викладач може 
керувати кількома студентами. 

Види запитів в інформаційний системі:
\begin{enumerate}
    \item Отримати перелік і загальне число студентів зазначених груп 
    або вказаного курсу (курсів) факультету повністю, 
    за статевою ознакою, року, віком, ознакою наявності дітей, 
    за ознакою отримання і розміром стипендії. 
    \item Отримати список і загальне число викладачів зазначених кафедр 
    або зазначеного факультету повністю або зазначених категорій 
    (асистенти, доценти, професори і т.д.) за статевою ознакою, 
    року, віком, ознакою наявності та кількості дітей, розміру 
    заробітної плати, є аспірантами, захистили кандидатські, 
    докторські дисертації в зазначений період.
    \item Отримати перелік і загальне число тем кандидатських 
    і докторських дисертацій, які захистили співробітники 
    зазначеної кафедри для зазначеного факультету.
    \item Отримати перелік кафедр, які проводять заняття 
    у зазначеній групі або на зазначеному курсі 
    вказаного факультету в зазначеному семестрі, 
    або за вказаний період.
    \item Отримати список і загальне число викладачів, 
    які проводили (проводять) заняття по вказаній 
    дисципліні в зазначеній групі або на зазначеному 
    курсі вказаного факультету.
    \item Отримати перелік і загальне число викладачів, 
    які проводили (проводять) лекційні, семінарські 
    та інші види занять у зазначеній групі або на 
    зазначеному курсі вказаного факультету в зазначеному 
    семестрі, або за вказаний період.
    \item Отримати список і загальне число студентів 
    зазначених груп, які здали залік або іспит
    з вказаною дисципліни зі встановленою оцінкою.
    \item Отримати список і загальне число студентів 
    зазначених груп або вказаного курсу зазначеного 
    факультету, які здали зазначену сесію на відмінно, 
    без трійок, без двійок.
    \item Отримати перелік викладачів, які беруть (брали) 
    іспити в зазначених групах, із зазначених дисциплін, 
    в зазначеному семестрі.
    \item Отримати список студентів зазначених груп, 
    яким заданий викладач поставив деяку оцінку 
    за іспит з певних дисциплін, в зазначених семестрах, 
    за деякий період.
    \item Отримати список студентів і тим дипломних робіт 
    на зазначеній кафедрі або у зазначеного викладача.
    \item Отримати список керівників дипломних робіт 
    по заданій кафедрі або факультету повністю 
    і окремо по деяким категоріям викладачів.
    \item Отримати навантаження викладачів (назва дисципліни, кількість годин), 
    її обсяг на окремі види занять і загальне навантаження в зазначеному 
    семестрі для конкретного викладача або для викладачів зазначеної кафедри.
\end{enumerate}