% !TEX root = ../main.tex
\newpage
\likechapter{Вступ}

ЗВО (заклад вищої освіти) є досить складною системою, в якій непросто описати всі
сутності та зв'язки між ними. Основною задачею бази даних ЗВО є забезпечення зручного
та швидкого доступу до інформації про факультети та кафедри, викладачів та їх навантаження, студентів, розклад.
Необхідно враховувати реальну структуру закладу вищої освіти та зв'язки між наведеними сутностями.

\textbf{Актуальність.} Функціонування такої складної системи, як заклад вищої освіти,
неможливе без використання ефективної інформаційної системи, яка забезпечуватиме 
швидкий та зручний доступ до потрібної інформації.

\textbf{Мета.} Метою роботи є розробка автоматизованої інформаційної системи
для закладу вищої освіти, яка дозволить зберігати всю необхідну інформацію, забезпечить 
виконання всіх видів інформаційних запитів, які необхідні при експлуатації даної 
системи.

\textbf{Завдання.} Спроектувати базу даних та підготувати усі необхідні 
запити та процедури для роботи з нею.

\textbf{Практичне значення.} Вдосконалення навичок SQL-програмування,
аналізу предметної області, проектування баз даних.

\textbf{Програмне забезпечення.} При виконанні роботи було використано
СУБД MySQL 8.0, веб-інтерфейс реалізовано з використанням фреймворку Flask
мови програмування Python 3.9. Використовувалися ОС Windows 10, середовища розробки
Visual Studio Code і MySQL Workbench та система контролю версій Git.